\documentclass{article}
\usepackage{graphicx} % Required for inserting images
%\usepackage{mathrsfs}

\title{Comparison of Frank-Wolfe Varients for White-Box Adversarial Attacks}
\author{Tanner Aaron Graves - 2073559\and Alessandro Pala - 2107800}
\date{June 2024}

\begin{document}

\maketitle

\section{Introduction}
What are Adversarial attacks\\
Problem statement\\
Type of Norms

\section{Algorithims}
\subsection{Frank-Wolfe}

\subsection{Pairwise Frank-Wolfe}
\subsection{Away-Step Frank-Wolfe}
\section{Results}
Introduce Datasets
\subsection{Momentum}
\subsection{LMO vs. Linesearch}
I really don't want to implement linesearch. As it is a waste of time when LMO is cheap but if we need easy content we can do this section.
\subsection{$\epsilon$ Choice}
Create plot showing how accurate attacks are with different $\epsilon$ constraints.
\section{Convergence Analysis}
The constrained nature of the Adversarial Attack problem means that the norm of the gradient $||\nabla_x f(x)||$ is not a sutible convergence criterion as boundry points need not have $0$ gradient. 
The Frank-Wolfe gap provides provides measure of both optimality and point feasibility. It is a measure of the maximum improvement over the current iteration $x_t$ within the constraints $C$ and defined
$$g(x_t) = \max_{x\in C} \langle x - x_t, -\nabla f(x_t)\rangle$$
We always have $g(x_t) \geq 0$ and  its usefullness as a convergence criterion comes from $g(x_t) = 0$ iff $x_t$ is a stationary point. 
% See what the slides say about stationary points
For convex problems, we would have that the linear approximation $f(x_t) + \langle x_t - x, -\nabla f(x_t) \rangle \geq f(x)$. However, the loss of DNNs as commnly the subject of adversarial attacks, are highly non-convex, making this only true locally. This complicate the convergence of Frank-Wolfe in this application, but it is still gaurenteed.

\subsection{Frank-Wolfe}
\subsection{Pairwise Frank-Wolfe}
\subsection{Away-Step Frank-Wolfe}
\end{document}
